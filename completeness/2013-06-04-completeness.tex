\documentclass[]{article}
\usepackage[T1]{fontenc}
\usepackage{lmodern}
\usepackage{amssymb,amsmath}
\usepackage{ifxetex,ifluatex}
\usepackage{fixltx2e} % provides \textsubscript
% use upquote if available, for straight quotes in verbatim environments
\IfFileExists{upquote.sty}{\usepackage{upquote}}{}
\ifnum 0\ifxetex 1\fi\ifluatex 1\fi=0 % if pdftex
  \usepackage[utf8]{inputenc}
\else % if luatex or xelatex
  \usepackage{fontspec}
  \ifxetex
    \usepackage{xltxtra,xunicode}
  \fi
  \defaultfontfeatures{Mapping=tex-text,Scale=MatchLowercase}
  \newcommand{\euro}{€}
\fi
% use microtype if available
\IfFileExists{microtype.sty}{\usepackage{microtype}}{}
\ifxetex
  \usepackage[setpagesize=false, % page size defined by xetex
              unicode=false, % unicode breaks when used with xetex
              xetex]{hyperref}
\else
  \usepackage[unicode=true]{hyperref}
\fi
\hypersetup{breaklinks=true,
            bookmarks=true,
            pdfauthor={},
            pdftitle={},
            colorlinks=true,
            urlcolor=blue,
            linkcolor=magenta,
            pdfborder={0 0 0}}
\urlstyle{same}  % don't use monospace font for urls
\setlength{\parindent}{0pt}
\setlength{\parskip}{6pt plus 2pt minus 1pt}
\setlength{\emergencystretch}{3em}  % prevent overfull lines
\setcounter{secnumdepth}{0}

\author{}
\date{}

\begin{document}

\section{Hacker School Analysis Seminar - Useful Sets and The
Completeness Axiom}

\subsection{Attribution}

Much of the content is closely adapted from
\href{http://books.google.com/books/about/Elementary_Analysis.html?id=ZDaSnKr_k5sC}{Elementary
Analysis: The Theory of Calculus} by Kenneth Ross. This book, or others
on mathematical analysis, should be consulted for further inquiry.

\subsection{Natural Numbers, Integers, and Fractions}

As we move forward, we will make use of a few different sets of numbers,
most of which you are probably already familiar with.

\[ \mathbb{N} = \{1,2,3,...\} \]

These are all the positive whole numbers, called the Natural Numbers.
Sometimes $0$ is included, but to stick with the notation in the Ross
text, we will exclude it.

\[ \mathbb{Z} = \{\pm n \forall n \in \mathbb{N} \} \cup \{0\} \]

This reads in plain english as ``plus and minus n for all n in
$\mathbb{N}$ union the set containing 0''.

\[ \mathbb{Q} = \{\frac{p}{q} \forall p,q \in \mathbb{Z} \mid q \ne 0 \} \]

These are more commonly known as fractions, (and maybe less commonly the
rational numbers) and include any number which can be written as the
division of any other two numbers of $\mathbb{Z}$.

\subsection{Real Numbers}

The final set, and the most important for our little seminar, is
$\mathbb{R}$. However, defining $\mathbb{R}$ is a bit tedious and beyond
our scope. Instead, we will describe it as the completion of
$\mathbb{Q}$. That is, it ``fill in the gaps'' that $\mathbb{Q}$ has,
such as $\pi$, $\sqrt{2}$, and $e$.

We will also introduce the Completeness Axiom, which assures us that
$\mathbb{R}$ has ``no gaps''. First, a couple definitions:

\subsubsection{Definition 1.1}

Let $S$ be a nonempty subset of $\mathbb{R}$.

(a). If $S$ contains a largest element $s_0$ (that is, $s_0 \in S$ and
$s \le s_0$ for all $s \in S$), then we call $s_0$ the \emph{maximum} of
$S$, i.e. $s_0 = \max S$.

(b). If $S$ contains a smallest element, then we call it the
\emph{minimum} of $S$, i.e. $\min S$.

\subsubsection{Definition 1.2}

Let $S$ be a nonempty subset of $\mathbb{R}$.

(a). If a real number $M$ satisfies $s \le M$ for all $s \in S$, then
$M$ is called the \emph{upper bound} of $S$ and the set $S$ is said to
be \emph{bounded above}.

(b). If a real number $m$ satisfies $m \le s$ for all $s \in S$, then
$m$ is called the \emph{lower bound} of $S$ and the set $S$ is said to
be \emph{bounded below}.

(c). The set $S$ is said to be \emph{bounded} if it is bounded above and
bounded below. Thus $S$ is bounded if there exists real numbers $m$ and
$M$ such that $S \subseteq [m,M]$.

\subsubsection{Notation Note}

\[ [m,M] = \{x \forall x \in \mathbb{R} \mid m \le x \le M \} \]
\[ (m,M) = \{x \forall x \in \mathbb{R} \mid m < x < M \} \]

\subsubsection{Definition 1.3}

Let $S$ be a nonempty subset of $\mathbb{R}$.

(a). If $S$ is bounded above and $S$ has a least upper bound, then we
will call it the \emph{supremum of} $S$ and denote it by $\sup S$.
Formally, $s_0 = \sup(S)$ iff (1). $s_0 \ge s$ for all $s in S$, and
(2). for any $t \in \mathbb{R}$ such that $t \ge s$ for all $s \in S$,
then $t \ge s_0$.

(b). If $S$ is bounded below and $S$ has a greatest lower bound, then we
will call it the \emph{infimum of} $S$ and denote it by $\inf S$.

\subsubsection{Examples}

(a). If a set $S$ has a maximum, then $\max S = \sup S$. Similarly, if
it has a minimum, then $\min S = \inf S$.

(b). If $a,b \in \mathbb{R}$ and $a < b$, then
\[ \sup [a,b] = \sup(a,b) = \sup[a,b) = \sup(a,b] = b \]

(c). inf $\mathbb{N}$ = 1

(d). If $A = \{r \in \mathbb{Q} \mid 0 \le r \le \sqrt{2} \}$, then
$\sup A = \sqrt{2}$ and $\inf A = 0$

\subsubsection{Completeness Axiom}

Every nonempty subset $S$ of $\mathbb{R}$ that is bounded above has a
least upper bound. In other words, $\sup S$ exists and is a real number.

Since this is an axiom, it is one of the basic assumptions that we are
making in order to conduct out analysis. We are also making a number of
other assumptions about things like set theory, but we won't go into
that detail here. This will be the most important ``tool'' we will use,
and most of what we will learn over the next few weeks would not be true
without the completeness axiom.

\subsubsection{Corollary}

Every nonempty set $S$ of $\mathbb{R}$ that is bounded below has a
greatest lower bound, i.e. $\inf S$ exists.

\subsubsection{Proof (Our first one!)}

Let $S$ be bounded below, and let $-S$ be the set $\{-s : s \in S\}$.
Since $S$ is bounded below, by definition there exists an $m$ in
$\mathbb{R}$ such that $m \le s$ for all $s \in S$. This implies that
$-m \ge -s$ for all $s \in S$, and thus $-m \ge u$ for all $u \in -S$.
Thus $-S$ is bounded above by $-m$, and by the Completeness Axiom,
$\sup(-S)$ exists.

Let $s_0 = -\sup(-S)$. We claim that $s_0$ is $\inf(S)$, which means
that we need to show:

(1). $s_0 \le s$ for all $s \in S$

(2). for any $t$, if $t \le s$ for all $s \in S$, then $t \le s_0$

To prove (1), note that $-s_0 = \sup(-S)$, and by definition,
$\sup(-S) \ge u$ for all $u \in -S$. Replacing, we get $-s_0 \ge u$ for
all $u \in -S$. Multiplying by $-1$, we get $s_0 \le -u$ for all
$u \in -S$, thus $s_0 \le s$ for all $s in S$, as desired.

To prove (2), suppose by way of contradiction that there exists a $t$
such that $t \le s$ for all $s \in S$, but $t > s_0$. Then, $-t \ge -s$
for $s \in S$, which implies $-t \ge u$ for all $u \in -S$. Similarly,
$-t < -s_0$. However, this contradicts the fact that $-s_0 = \sup(-S)$.
$\mathbb{QED}$.

\end{document}
