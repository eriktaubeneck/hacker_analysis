\documentclass[]{article}
\usepackage[T1]{fontenc}
\usepackage{lmodern}
\usepackage{amssymb,amsmath}
\usepackage{ifxetex,ifluatex}
\usepackage{fixltx2e} % provides \textsubscript
% use upquote if available, for straight quotes in verbatim environments
\IfFileExists{upquote.sty}{\usepackage{upquote}}{}
\ifnum 0\ifxetex 1\fi\ifluatex 1\fi=0 % if pdftex
  \usepackage[utf8]{inputenc}
\else % if luatex or xelatex
  \usepackage{fontspec}
  \ifxetex
    \usepackage{xltxtra,xunicode}
  \fi
  \defaultfontfeatures{Mapping=tex-text,Scale=MatchLowercase}
  \newcommand{\euro}{€}
\fi
% use microtype if available
\IfFileExists{microtype.sty}{\usepackage{microtype}}{}
\ifxetex
  \usepackage[setpagesize=false, % page size defined by xetex
              unicode=false, % unicode breaks when used with xetex
              xetex]{hyperref}
\else
  \usepackage[unicode=true]{hyperref}
\fi
\hypersetup{breaklinks=true,
            bookmarks=true,
            pdfauthor={},
            pdftitle={},
            colorlinks=true,
            urlcolor=blue,
            linkcolor=magenta,
            pdfborder={0 0 0}}
\urlstyle{same}  % don't use monospace font for urls
\setlength{\parindent}{0pt}
\setlength{\parskip}{6pt plus 2pt minus 1pt}
\setlength{\emergencystretch}{3em}  % prevent overfull lines
\setcounter{secnumdepth}{0}

\author{}
\date{}

\begin{document}

\section{Hacker School Analysis Seminar - Sequences}

\subsection{Attribution}

Much of the content is closely adapted from
\href{http://books.google.com/books/about/Elementary_Analysis.html?id=ZDaSnKr_k5sC}{Elementary
Analysis: The Theory of Calculus} by Kenneth Ross. This book, or others
on mathematical analysis, should be consulted for further inquiry.

\subsection{Useful Properties}

\subsubsection{Triangle Inequality}

\subsubsection{Claim:}

\emph{$\vert a + b\vert  \le \vert a\vert  + \vert b\vert $ for all
$a,b \in \mathbb{R}$.}

\paragraph{Proof:}

See Ross text.

\subsubsection{Archimedean Property}

\paragraph{Claim:}

\emph{If $a>0$ and $b>0$, then there exists $n \in \mathbb{N}$ such that
$na > b$.}

In essence, this states that no matter how small of an $a$ and how large
of a $b$ we choose, we can always find an integer multiple of $a$ that
will exceed $b$.

\paragraph{Proof:}

Suppose, by way of contradiction, that the Archimedean property fails.
Then there exists $a>0$ and $b>0$ such that $na \le b$ for all
$n in \mathbb{N}$. By definition, $b$ is then an upper bound on the set
$S = \{na : n \in \mathbb{N}\}$.

Using the completeness axiom, let $s_0 = \sup S$. Since $a > 0$, we have
$s_0 < s_0 + a$, so $s_0 - a < s_0$. Since $s_0$ is the least upper
bound of $S$, $s_0 - a$ cannot be an upper bound of $S$. This implies
that there exists an $s = n_0 a \in S$ such that $s_0 - a < n_0 a$ for
some $n_0 \in \mathbb{N}$. Adding $a$ to each side of the inequality, we
get $s_0 < (n_0 + 1) a$. Since $n_0 + 1 \in \mathbb{N}$,
$(n_0 +1) a \in S$, and so $s_0$ is not an upper bound for $S$.
Contradiction. $\mathbb{QED}$.

\subsubsection{Denseness of $\mathbb{Q}$}

\paragraph{Claim:}

\emph{If $a,b \in \mathbb{R}$ and $a<b$, then there exists a rational
number $r \in \mathbb{Q}$ such that $a < r < b$.}

\paragraph{Pseudo-Proof:}

We need to show that $a < \frac{m}{n} < b$ for some
$m,n \in \mathbb{N}$, where $n > 0$. Thus, we need

\[ an < m < bn \]

Since $b-a > 0$, the Archimedean property shows that there exists an
$n \in \mathbb{N}$ such that $n(b-a) > 1$. Thus, $bn - an > 1$, and so
it is \emph{fairly evident} that there exists an $m$ between $an$ and
$bn$, so our claim holds. (Actually \emph{proving} that this $m$ exists
is a bit more delicate, see page 24/25 in Ross for more detail.)

\subsection{Sequences}

Formally, a \emph{sequence} is a function from
$\{ n \in \mathbb{Z}: n \ge m\}$, where $m$ is usually 0 or 1, into
$\mathbb{R}$. However, it is customary to write $s_n$ rather than
$s(n)$, and is also convenient to write $(s_n)_{n=m}^{\infty}$,
$(s_m, s_{m+1}, s_{m+2})$, or, when $m = 1$, $(s_n)_{n \in \mathbb{N}}$.

\subsubsection{Examples (1)}

(a). Consider the sequence $(a_n)_{n \in \mathbb{N}}$ where
$a_n = \frac{1}{n^2}$. This is the sequence
$(1, \frac{1}{4}, \frac{1}{9}, \frac{1}{16}, \frac{1}{25}, ...)$.
Formally this is the function with domain $\mathbb{N}$ with value
$\frac{1}{n^2}$ for each $n$. The \emph{set} of values is
$\{1, \frac{1}{4}, \frac{1}{9}, \frac{1}{16}, \frac{1}{25}, ...\}$.

(b). Consider the sequence $(b_n)_{n = 0}^{\infty}$ where
$b_n = (-1)^n$. The sequence is $(1,-1,1,-1,...)$, however its
\emph{set} of values is $\{-1,1\}$.

It's important to distinguish between a sequence and its set of values,
and we will always use parentheses $( \ )$ to signify a sequence and
braces $\{ \ \}$ to signify a set.

(c). Consider the sequence $(c_n)_{n \in \mathbb{N}}$ where
$c_n = (1 + \frac{1}{n})^n$. This is the sequence
$(2, (\frac{3}{2})^2, (\frac{4}{3})^3, (\frac{5}{4})^4, ...)$, or
approximately

\[ (2, 2.25, 2.3704, 2.4414, 2.4883, 2.5216, 2.5465, 2.5658, ...). \]

$c_{100}$ is approximately 2.7048, and $c_{1000}$ is approximately
2.7169.

(d.) Consider the sequence $(d_n)_{n \in \mathbb{N}}$ where
$d_n = \pi^{n-1}$. This is the sequence
$(\pi^0, \pi^1, \pi^2, \pi^3, ...)$, or approximately

\[ (0, 3.1416, 9.8696, 31.0063, ...). \]

$d_{100}$ is approximately $5.1878 \times 10^{49}$ and $d_{1000}$ causes
an OverflowError in Python.

\subsubsection{Limits}

The \emph{limit} of a sequence $(s_n)$ is a real number which the values
$s_n$ are ``close'' to for large values of $n$. In example (a), the
values are ``close'' to 0 for large $n$, and in example (c), the values
are close to Euler's constant, $e$, for large $n$. However, example (b)
doesn't seem to get close to any number, but instead jumps between $-1$
and $1$. As we'll see in the following definition, a \emph{limit} will
require the sequence values to be close to the limit value for
\emph{all} large $n$, so neither $1$ or $-1$ will be limits of $(c_n)$.

\paragraph{Definition (1):}

A sequence $(s_n)$ of real numbers is said to \emph{converge} to the
real number $s$ if and only if for every $\epsilon > 0$ there exists a
number $N$ such that $\vert s_n - s\vert  < \epsilon$ for all $n > N$.

\paragraph{Definition (2):}

A sequence $(s_n)$ of real numbers is said to \emph{diverge towards
$\infty$} if and only if for every $M \in \mathbb{R}$ there exists a
number $N$ such that $s_n > M$ for all $n > N$.

\paragraph{Definition (3):}

A sequence $(s_n)$ of real numbers is said to \emph{diverge towards
$-\infty$} if and only if for every $M \in \mathbb{R}$ there exists a
number $N$ such that $s_n < M$ for all $n > N$.

\subsubsection{Examples (2)}

We will now prove a couple of the above examples (1).

(a.) We aim to prove that $\lim_{n \to \infty} a_n = 0$, so we must show
that there exists $N in \mathbb{N}$ such that for all $n >N$,
$\vert a_n - 0\vert  < \epsilon$ for all $\epsilon > 0$.

Let $\epsilon > 0$ be given. Now, define
$N = \left \lceil \frac{1}{\sqrt{\epsilon}} \right \rceil $. Then for
all $n > N$,

\[\vert a_n - 0\vert  = a_n = \frac{1}{n^2} < \frac{1}{N^2} = \frac{1}{\left \lceil \frac{1}{\sqrt{\epsilon}} \right \rceil^2} \le \frac{1}{\left (\frac{1}{\sqrt{\epsilon}} \right )^2} = \frac{1}{\left (\frac{1^2}{\sqrt{\epsilon}^2} \right )} = \left( \sqrt{\epsilon} \right)^2 = \epsilon.\]

$\mathbb{QED}$.

The relevant function (in python) would be:

\begin{verbatim}
from math import sqrt, ceil
def example_a(epsilon):
    N = int(ceil(sqrt(epsilon)))
    return N
\end{verbatim}

(d.) We aim to prove that $\lim_{n \to \infty} d_n$ diverges towards
$\infty$, so we must show that for all $M \in \mathbb{R}$ their exists
$N \in \mathbb{N}$ such that $d_n > M$ for all $n > N$.

Let $M \in \mathbb{R}$ be given, and note that if $M \le 0$ and we show
that $d_n > 0$ for all $n \in \mathbb{N}$, then $d_n > M$ as well. So
let $\hat{M} = \max(M,1)$, and define
$N = \lceil \log_{\pi} \hat{M} \rceil $. Then for all $n > N$,

\[ d_n = \pi^n > \pi^N = \pi^{\lceil \log_{\pi} \hat{M} \rceil} \ge \pi^{ \log_{\pi} \hat{M} } = \hat{M} =  \max(M,1) \ge M. \]

$\mathbb{QED}$.

The relevant function (in python) would be:

\begin{verbatim}
from math import pi, log, ceil
def example_d(M):
    M = max(1,M)
    N = int(ceil(log(M,pi)))
    return N
\end{verbatim}

\subsubsection{Exercises}

\begin{enumerate}
\def\labelenumi{(\arabic{enumi})}
\itemsep1pt\parskip0pt\parsep0pt
\item
  Write out the first five terms of the following sequences.
\end{enumerate}

(a). $a_n  = \frac{n}{n+1}$

(b). $b_n = 2^{-n}$

(c). $c_n = n!$

(d). $d_n = 1 + \frac{2}{n}$

(e). $e_n = \frac{6n^2+7}{4n^2 - 9}$

(f). $f_n = (-1)^n n$

(g). $g_n = \frac{n^3+4}{7n^2 - 13}$

(h). $h_n = \sin(n \pi)$

(i). $i_n = \frac{1}{n} \cos(n)$

(j). $j_n = \cos(n)$

\begin{enumerate}
\def\labelenumi{(\arabic{enumi})}
\setcounter{enumi}{1}
\item
  For each sequence above, determine whether it converges, diverges to
  $\pm \infty$, or doesn't converge/diverge, and if so give it's limit.
  Proofs are not required.
\item
  For each sequence above that converges (or diverges to $\pm \infty$),
  find the mathematical function for $N$ given $M$ or $\epsilon$.
\item
  For each sequence above that converges (or diverges to $\pm \infty$),
  write a function in a programming language of your choice for $N$
  given $M$ or $\epsilon$.
\item
  For each sequence above that converges (or diverges to $\pm \infty$),
  prove that it converges to the limit (or diverges to $\pm \infty$).
\end{enumerate}

\end{document}
